\documentclass[aspectratio=43]{beamer}
\usepackage{ragged2e}
\usepackage{multirow}
\usepackage{alltt}

\usetheme{CSCS}


\newcommand{\SummerSchoolYear}{2016}
\newcommand{\SummerSchoolDate}{July 20 -- 21}
\newcommand{\SummerSchoolAuthor}{Maxime Martinasso}

\newcommand{\footlinetext}{Summer School \SummerSchoolYear{} -- MPI}

\author{\SummerSchoolAuthor, CSCS}
\title{Message Passing Interface (MPI)}
\subtitle{Summer School \SummerSchoolYear{}  -- Effective High Performance Computing}
\date{\SummerSchoolDate, \SummerSchoolYear}


% Select the image for the title page
%\newcommand{\picturetitle}{cscs_images/image3.pdf}
\newcommand{\picturetitle}{cscs_images/image5.pdf}
%\newcommand{\picturetitle}{cscs_images/image6.pdf}

\begin{document}

% TITLE SLIDE
\cscstitle

\begin{frame}{Course Objectives}
\begin{itemize}
\item Hybrid OpenMP/MPI, preparing MPI for OpenMP
\item One sided MPI communication
\end{itemize}
\end{frame}

% TABLE OF CONTENT SLIDE
\cscstableofcontents[hideallsubsections]{General Course Structure}

\section{An introduction to MPI}
\section{Point-to-point communications}
\section{Collective communications}
\section{Topology}
\section{Datatypes}
\section{Other topics}
\cscstableofcontents[currentsection]{General Course Structure}

% CHAPTER SLIDE
\cscschapter{Configure MPI to enable OpenMP}

\subsection{Hybrid OpenMP/MPI}

\begin{frame}[fragile]{Why OpenMP with MPI}
\begin{itemize}
\item Code not fully parallel with MPI can benefit of extra threads
\item Memory limit the number of MPI ranks per node
\item To increase scalability (but you may well be not faster than with MPI alone)
\end{itemize}
\end{frame}

\begin{frame}[fragile]{Preparing MPI for OpenMP}
MPI requires to be setup with threads enabled:
\begin{itemize}
\item \lstinlinePseudo{MPI_Init} should be replaced by \lstinlinePseudo{MPI_Init_thread}
\end{itemize}
\begin{Pseudolisting}[]{}
MPI_Init_thread(required, provided, ierror)
\end{Pseudolisting}

\lstinlinePseudo{required} specifies the requested level of thread support, and the actual level of support is then returned into \lstinlinePseudo{provided}.\\
You should check the value of \lstinlinePseudo{provided} after the call
\end{frame}

\begin{frame}[fragile]{MPI Thread level}
\begin{itemize}
    \item \lstinlinePseudo{MPI_THREAD_SINGLE}: only one thread (MPI-only application)
    \item \lstinlinePseudo{MPI_THREAD_FUNNELED}: Only master thread will make threads (master = the thread calling \lstinlinePseudo{MPI_Init_thread})
    \item \lstinlinePseudo{MPI_THREAD_SERIALIZED}: Only one thread at a time will call MPI (user responsibility)
    \item \lstinlinePseudo{MPI_THREAD_MULTIPLE}: Any thread may call MPI at any time, however that leads to slower performance (lock mechanism in MPI)
\end{itemize}
\end{frame}


% CHAPTER SLIDE
\cscschapter{One sided communication}

\subsection{One sided communication}

\begin{frame}[fragile]{No more receives?}
\begin{itemize}
    \item A memory region is exposed by all ranks (window)
    \item Each rank can read or write in a region
    \item Transfers and synchronisation are decoupled
\end{itemize}
\end{frame}

\begin{frame}[fragile]{Terminology}
\begin{itemize}
\item Origin process: Process with the source buffer, initiates the operation
\item Target process: Process with the destination buffer
\item Epoch: Virtual time where operations are not completed. Data is consistent after new epoch is started.
\item Ordering: order of message between two processes (can be relaxed compared to a strict order)
\end{itemize}

\end{frame}

\begin{frame}[fragile]{One sided communication usage}
\begin{itemize}
\item Expose memory collectively
\item Allocate exposed memory
\item Dynamic memory exposure
\item Communication (put, get, rput, rget)
\item Accumulate (acc, racc, get\_acc, rget\_acc, fetch\&op, cas)
\item Synchronization
\end{itemize}
\end{frame}

\begin{frame}[fragile]{Expose memory}

\begin{Pseudolisting}[]{}
MPI_Win_create(ptr, MPI_Aint size, disp_unit, MPI_Info info,
               MPI_Comm comm, MPI_Win win)
MPI_Win_allocate(MPI_Aint size, disp_unit, MPI_Info info,
                 MPI_Comm comm, ptrptr, MPI_Win win)
MPI_Win_free(MPI_Win win)
\end{Pseudolisting}
Exposes (and allocate) the memory buffer to other ranks.\\
It is a collective operations among all ranks.\\

\lstinlinePseudo{MPI_Info} contains user configuration.\\
\lstinlinePseudo{MPI_Aint} type that holds any valid address.
\end{frame}

\begin{frame}[fragile]{Dynamic memory}
It is possible to create a window with no memory attached.

\begin{Pseudolisting}[]{}
MPI_Win_create_dynamic(MPI_Info info, MPI_Comm comm,
                       MPI_Win win)
\end{Pseudolisting}

It is possible to attach/detach memory to/from a window.
\begin{Pseudolisting}[]{}
MPI_Win_attach(MPI_Win win, ptr, MPI_Aint size)
MPI_Win_detach(MPI_Win win, ptr)
\end{Pseudolisting}
\end{frame}

\begin{frame}[fragile]{Communication}
Two functions:
Write data into the window (put)
\begin{Pseudolisting}[]{}
MPI_Put(origin_addr, origin_count, origin_datatype, 
        target_rank, target_disp, target_count, target_datatype, win)
\end{Pseudolisting}

Read data from the window (get)
\begin{Pseudolisting}[]{}
MPI_Get(origin_addr, origin_count, origin_datatype, 
        target_rank, target_disp, target_count, 
        target_datatype, win)
\end{Pseudolisting}

\begin{itemize}
\item Non blocking communication
\item Concurrent accesses give undefined behaviour (not an error).
\end{itemize}

\end{frame}


\begin{frame}[fragile]{Accumulate}
\begin{Pseudolisting}[]{}
MPI_Accumulate(origin_addr, origin_count, origin_datatype,
               target_rank, target_disp, target_count, 
               target_datatype, op, win)
\end{Pseudolisting}

Remote accumulations, replace value in target buffer with accumulated.

\begin{itemize}
\item Only predefined operations
\item Conflicting accesses are managed by the selected ordering
\end{itemize}

Other operations: \lstinlinePseudo{MPI_Fetch_and_op, MPI_Compare_and_swap}
\end{frame}

\begin{frame}[fragile]{Synchronisation}
\begin{Pseudolisting}[]{}
MPI_Win_fence(assert, win)
\end{Pseudolisting}
Collectively synchronizes all RMA calls started before fence on the window.

Other synchronisation:
\begin{itemize}
\item Asynchronous with PSCW:\@ Post, Start, Complete, Wait
\item Lock / Unlock, Lock All
\item Flush outstanding operations
\item Synchronise a window
\end{itemize}
\end{frame}

% THANK YOU SLIDE
\cscsthankyou{Thank you for your attention.}

\end{document}
